\documentclass{article}
\usepackage{listings}
\usepackage{graphicx}
\usepackage{hyperref}
\usepackage{amsmath}
\usepackage{geometry}
\usepackage[utf8]{inputenc}

\geometry{margin=1in}
\title{ETO-HEOM Tutorial}

\begin{document}

\maketitle

\section*{Quick Start: Workflow Summary}

This section provides a summary of the full workflow. Follow these steps to run a full 2DES simulation using ETO-HEOM:

\begin{enumerate}
    \item \textbf{Build the Solver}  
    Modify the compiler and linking path to fit your sustem and compile the CPU and/or GPU solvers:
    \begin{lstlisting}[language=bash]
    cd /path/to/ETO-HEOM/
    make
    \end{lstlisting}

    \item \textbf{Initialize Working Directory}  
    Use the setup script to create a simulation workspace:
    \begin{lstlisting}[language=bash]
    ./setup_cpu_job.sh JOBNAME BATHTYPE
    # or
    ./setup_gpu_job.sh JOBNAME BATHTYPE
    \end{lstlisting}

    \item \textbf{Edit Template Input File}  
    Modify \texttt{key.key-tmpl} to set up:
    \begin{itemize}
        \item System size, HEOM level, Hamiltonian, Disorder
        \item Dipole directions, pulse settings, and time domain
    \end{itemize}

    \item \textbf{Generate Bath Parameters}  
    Automatically fill in the bath section by executing:
    \begin{lstlisting}[language=bash]
    python3 {BATHTYPE}__ETOM.py
    # or run {BATHTYPE}__ETOM_example.ipynb
    \end{lstlisting}

    \item \textbf{Generate Input Files}  
    Configure and run the input generation script:
    \begin{lstlisting}[language=bash]
    ./gen_2d_data.sh
    \end{lstlisting}

    \item \textbf{Submit PBS Jobs}  
    Submit the simulation jobs to the cluster:
    \begin{lstlisting}[language=bash]
    ./submit_jobs.sh
    \end{lstlisting}

    \item \textbf{Wait Until Completion}  
    Track your jobs with \texttt{qstat} or log files. Ensure all output files are generated in \texttt{./2d-output}.

    \item \textbf{Analyze Results and Generate 2D Spectrum}  
    After all jobs finish, analyze the results:
    \begin{lstlisting}[language=bash]
    python3 gen_2d_spectrum.py
    # or run gen_2d_spectrum_example.ipynb
    \end{lstlisting}
\end{enumerate}

\vspace{1em}
For detailed instructions, continue reading the full tutorial below.

\newpage

\section{Build Instructions}

Before setting up any jobs, first check the compiler configuration to fit your system.
You can modify the following path in the Makefile if needed:

\begin{lstlisting}[language=bash]
    # C++ compiler and flags
    CXX := g++
    
    # CUDA compiler and flags
    NVCC := /path/to/cuda-XX.X/bin/nvcc
    
    # Include and library paths
    INCLUDES := -I/usr/include
    CUDA_INC := -I/path/to/cuda-XX.X/include
    CUDA_LIB := -L/path/to/cuda-XX.X/lib64
\end{lstlisting}
Now you can compile the ETO-HEOM using the unified Makefile in the project root:
\begin{lstlisting}[language=bash]
    cd /path/to/ETO-HEOM/
    make
\end{lstlisting}
If you only want to build one version:
\begin{lstlisting}[language=bash]
    make cpu        # Build only CPU version
    make gpu        # Build only GPU version
\end{lstlisting}
To clean all build artifacts:
\begin{lstlisting}[language=bash]
    make clean      # Remove all object files and binaries
\end{lstlisting}

\section{Job Setup Scripts}

Before using any setup scripts, ensure that the \texttt{HOME\_PATH} variable in the script files is set to the current path of your \texttt{ETO-HEOM} project.
\begin{lstlisting}[language=bash]
    HOME__PATH = /path/to/ETO-HEOM/
\end{lstlisting}


\subsection{CPU Job Setup}

Initializes a simulation folder for CPU-based 2DES simulations.

\subsubsection*{Usage}
\begin{lstlisting}[language=bash]
./setup_cpu_job.sh JOBNAME BATHTYPE
\end{lstlisting}

\subsubsection*{Arguments}
\begin{itemize}
    \item \texttt{JOBNAME}: Custom identifier for the simulation (e.g., \texttt{dimer})
    \item \texttt{BATHTYPE}: Type of spectral density — must be one of:
    \begin{itemize}
        \item \texttt{debye\_lorentz}
        \item \texttt{ohmic}
        \item \texttt{superohmic}
    \end{itemize}
\end{itemize}

\subsubsection*{Resulting Structure}
\begin{verbatim}
    CPU_JOBNAME_BATHTYPE/
    |-- 2d-input/                    # Input .key files
    |-- 2d-output/                   # Output .out files
    |-- pbs-script/                  # PBS scripts
    |   |-- pbserr/                  # PBS error logs
    |   |-- pbslog/                  # PBS output logs
    |-- key.key-tmpl                 # Input template file
    |-- gen_2d_data.sh               # Script to generate .key files
    |-- clean_2d_data.sh             # Script to clean .key and .out files
    |-- gen_2d_spectrum.py           # Script to gen 2d spectrum
    |-- gen_2d_spectrum_example.py   # gen 2d spectrum example
    |-- BATHTYPE_ETOM.py             # ETO model for the specified bath type
    |-- BATHTYPE_ETOM_example.ipynb  # ETO model example
    |-- submit_jobs.sh               # Script to generate & submit PBS jobs (CPU)
    |-- README.md                    # Usage of working file
\end{verbatim}

\subsection{GPU Job Setup}

Initializes a simulation folder for GPU-based 2DES simulations.

\subsubsection*{Usage}
\begin{lstlisting}[language=bash]
./setup_gpu_job.sh JOBNAME BATHTYPE
\end{lstlisting}

\subsubsection*{Arguments}
\begin{itemize}
    \item \texttt{JOBNAME}: Custom identifier for the simulation (e.g., \texttt{dimer})
    \item \texttt{BATHTYPE}: Type of spectral density — must be one of:
    \begin{itemize}
        \item \texttt{debye\_lorentz}
        \item \texttt{ohmic}
        \item \texttt{superohmic}
    \end{itemize}
\end{itemize}

\subsubsection*{Resulting Structure}
\begin{verbatim}
    GPU_JOBNAME_BATHTYPE/
    |-- 2d-input/                    # Input .key files
    |-- 2d-output/                   # Output .out files
    |-- pbs-script/                  # PBS scripts
    |   |-- pbserr/                  # PBS error logs
    |   |-- pbslog/                  # PBS output logs
    |-- key.key-tmpl                 # Input template file
    |-- gen_2d_data.sh               # Script to generate .key files
    |-- clean_2d_data.sh             # Script to clean .key and .out files
    |-- gen_2d_spectrum.py           # Script to gen 2d spectrum
    |-- gen_2d_spectrum_example.py   # gen 2d spectrum example
    |-- BATHTYPE_ETOM.py             # ETO model for the specified bath type
    |-- BATHTYPE_ETOM_example.ipynb  # ETO model example
    |-- submit_jobs.sh               # Script to generate & submit PBS jobs (GPU)
    |-- README.md                    # Usage of working file
\end{verbatim}

\newpage

\section{Template Input Setup}

After setting up your job folder, you must modify the \texttt{key.key-tmpl} file located in the working directory. This is the template input file that defines all system-specific parameters for the simulation.
Update the following sections to reflect your system configuration:

\subsection*{SIZE}
Define the system size including the ground and first excited states.

\subsection*{HEOM}
Define the HEOM configuration:
\begin{itemize}
    \item Format: \texttt{(SITE\_NUMBER) (TRUNCATION\_LEVEL)}
\end{itemize}

\subsection*{HAMILTONIAN}
Define the system Hamiltonian, including the ground and excited states. All values are in cm\textsuperscript{-1}.

\subsection*{DISORDER}
Static disorder matrix of the Hamiltonian, also in cm\textsuperscript{-1}.  
\textbf{First row:} number of samples and random seed.

\subsection*{BATH}
Bath information, automatically generated by \texttt{\{BATHTYPE\}\_ETOM.py}.  
\textit{You do not need to modify this section.}

\subsection*{DIPOLE}
\begin{itemize}
    \item First row: Number of transition dipole vectors, usually same as \texttt{(SITE\_NUMBER)}
    \item Following rows: Dipole direction and amplitude in XYZ components
\end{itemize}

\subsection*{POLARIZATION}
Define the polarization angles for the four pulses.  
\textit{You typically do not need to modify this section.}

\subsection*{PULSE}
\begin{itemize}
    \item First row: Number of pulses (currently only supports 3)
    \item Next 3 rows: Amplitude (\(< 10\) cm\textsuperscript{-1}), central time (fs), width (fs), and frequency (cm\textsuperscript{-1})
    \item Use placeholders \texttt{TAU1}, \texttt{TAU2}, \texttt{TAU3} that will be replaced automatically by \texttt{gen\_2d\_data.sh}
\end{itemize}

\subsection*{TIME}
Simulation time settings:
\begin{itemize}
    \item Format: \texttt{start time}, \texttt{end time (T\_END)}, \texttt{time step}, and \texttt{sampling interval}
    \item \texttt{T\_END} will be replaced dynamically during \texttt{gen\_2d\_data.sh} execution
\end{itemize}

\newpage

\section{Input File Generation Script}

\subsection*{Modify \texttt{gen\_2d\_data.sh}}

This shell script controls how the 2DES input files are generated. You may adjust the following key parameters at the top of the script:

\begin{itemize}
    \item \texttt{t0} — Initial central time for the first pulse (in fs). It is recommanded to set \texttt{t0} to $2 * pulse\ width + 10$, ensuring it cover the pulse envelop.
    \item \texttt{propagate\_time} — Duration to propagate after the last pulse (in fs). 600 fs is recommanded.
    \item \texttt{tau\_step} — Step size for coherent time between different files. (in fs) 10 fs is recommanded.
    \item \texttt{tau\_bound} — Upper and lower bounds for coherent time (in fs, \texttt{-tau\_bound} to \texttt{tau\_bound})
    \item \texttt{T=""} — list of population times \(T\) (in fs). You can add as more values as needed.
    \item \texttt{input\_file="key.key-tmpl"} — Path to the input template file
\end{itemize}

\subsubsection*{Script Functionality}

The script will:

\begin{itemize}
    \item Loop over values of coherent time from \(-\texttt{tau\_bound}\) to \(\texttt{tau\_bound}\)
    \item Automatically compute \texttt{TAU1}, \texttt{TAU2}, \texttt{TAU3}, and \texttt{T\_END} based on \texttt{t0}, \texttt{tau\_step}, and each value in \texttt{T}
    \item Replace placeholders in the template file with computed values
    \item Output input files as: \texttt{./2d-input/key\_\{tau2\}\_\{T\}.key}
\end{itemize}

You can modify this script to include more population times or to adjust the pulse timing logic as needed.

\subsection*{Run \texttt{gen\_2d\_data.sh}}

Once the script is configured, execute it to generate input files:

\begin{lstlisting}[language=bash]
./gen_2d_data.sh
\end{lstlisting}

The generated input files will be saved in the \texttt{./2d-input} directory.

\subsection*{Clean Input Files: \texttt{clean\_2d\_data.sh}}

To remove all generated input files and reset the input directory, use:

\begin{lstlisting}[language=bash]
./clean_2d_data.sh
\end{lstlisting}

This will delete all files inside the \texttt{./2d-input/} folder.

\newpage

\section{Generate Bath Parameters}

After the working directory is created using the setup script, the bath type (\texttt{BATHTYPE}) is already specified and embedded into the directory structure. You do not need to manually modify or set the bath type again.

\subsection*{Execute \texttt{BATHTYPE\_ETOM.py}}

To generate bath parameters for your simulation, run the corresponding Python script inside your working directory. This script will automatically insert bath information into \texttt{key.key-tmpl}:

\begin{lstlisting}[language=bash]
python3 debye_lorentz_ETOM.py
\end{lstlisting}

This command calculates the ETO model parameters based on the specified spectral density type and updates the \texttt{BATH} section in the template file accordingly.

\subsection*{Use Jupyter Notebook: \texttt{BATHTYPE\_ETOM\_example.ipynb}}

You may also use the accompanying Jupyter notebook to interactively generate bath parameters:

\begin{itemize}
    \item Launch the notebook with:
    \begin{lstlisting}[language=bash]
    jupyter notebook debye_lorentz_ETOM_example.ipynb
    \end{lstlisting}
    \item Execute the notebook cells step-by-step to calculate and visualize the bath parameters.
\end{itemize}

This method is recommended if you wish to explore or fine-tune the bath model before applying it.

\newpage 

\section{Submit PBS Jobs}

After generating all input files, execute the submission script to generate and submit PBS job scripts.

\subsection*{Run \texttt{submit\_jobs.sh}}

From your working directory, run:

\begin{lstlisting}[language=bash]
./submit_jobs.sh
\end{lstlisting}

This script will automatically:
\begin{itemize}
    \item Create PBS job scripts for the CPU or GPU version based on your setup
    \item Divide the \(\tau_2\) range into manageable chunks (CPU only)
    \item Submit the generated PBS scripts using \texttt{qsub}
\end{itemize}

\subsection*{Important Parameters to Check}

Before execution, ensure the following variables in \texttt{submit\_jobs.sh} are properly set:

\begin{itemize}
    \item \texttt{HOME\_PATH} — Automatically set by the setup script
    \item \texttt{JOBNAME} — Automatically set by the setup script
    \item \texttt{CPU\_2DES} or \texttt{GPU\_2DES} — Path to your binary executable
    \item \texttt{INPUT\_DIR} — Should be \texttt{./2d-input}
    \item \texttt{OUTPUT\_DIR} — Should be \texttt{./2d-output}
    \item \texttt{ERR\_DIR}, \texttt{LOG\_DIR} — Should point to subfolders inside \texttt{./pbs-script/}
    \item \texttt{TLIST} — Array of population times \(T\) (e.g., \texttt{0} or \texttt{(0 100 200)})
    \item \texttt{START\_TAU}, \texttt{END\_TAU}, \texttt{STEP\_TAU} — Define the \(\tau_2\) scan range
\end{itemize}

For the GPU submit script, it is necessary to setup cuda environment to access GPU correctly
\begin{lstlisting}[language=bash]
    export CUDA_HOME=/usr/local/cuda-XX.X
    export PATH=\$CUDA_HOME/bin:\$PATH
    export LD_LIBRARY_PATH=\$CUDA_HOME/lib64:\$LD_LIBRARY_PATH
\end{lstlisting}

\subsection*{PBS Script Generation: CPU Version}

For CPU-based simulations, the script:
\begin{itemize}
    \item Splits the \(\tau_2\) scan range into equal parts (\texttt{NUM\_SCRIPTS})
    \item Generates one PBS script per chunk under \texttt{pbs-script/}
    \item Submits them using \texttt{qsub}
\end{itemize}

Each PBS job:
\begin{itemize}
    \item Executes up to 11 parallel CPU processes (\texttt{ppn=11})
    \item Launches \texttt{CPU\_2DES} for each \texttt{key\_\{TAU\}\_\{T\}.key} input
    \item Outputs result to \texttt{2d-output/out\_\{TAU\}\_\{T\}.out}
\end{itemize}

\subsection*{PBS Script Generation: GPU Version}

For GPU-based simulations, the script:
\begin{itemize}
    \item Generates a single PBS job script under \texttt{pbs-script/}
    \item Sets GPU-specific resources (e.g., \texttt{nodes=gpu02}, CUDA environment variables)
\end{itemize}

The job:
\begin{itemize}
    \item Executes \texttt{GPU\_2DES} for each input file
    \item Logs results and errors in \texttt{pbs-script/pbslog} and \texttt{pbs-script/pbserr}
\end{itemize}

\subsection*{Output}

Submitted PBS jobs will automatically:
\begin{itemize}
    \item Read input files from \texttt{./2d-input}
    \item Write output files to \texttt{./2d-output}
    \item Write logs to \texttt{./pbs-script/pbslog/}
    \item Write errors to \texttt{./pbs-script/pbserr/}
\end{itemize}

Use \texttt{qstat}, \texttt{tail -f}, or PBS web portal to monitor job progress.

\section{Post-Processing: Generate 2D Spectrum}

After all PBS jobs have completed and the output files are available in the \texttt{./2d-output} directory, you can generate the 2D spectrum by running the analysis script.

\subsection*{Execute \texttt{gen\_2d\_spectrum.py}}

This script will parse the output files in \texttt{2d-output/}, perform 2D Fourier transforms over coherent times, and assemble the final 2D spectrum data.

To run the script:

\begin{lstlisting}[language=bash]
python3 gen_2d_spectrum.py
\end{lstlisting}

\noindent The resulting 2D spectrum will be saved in a format specified within the script. (e.g., png, svg, eps, etc.)

\subsection*{Use Jupyter Notebook: \texttt{gen\_2d\_spectrum\_example.ipynb}}

Alternatively, if you prefer interactive analysis or visualization, you can use the accompanying Jupyter notebook:

\begin{itemize}
    \item Launch the notebook with:
    \begin{lstlisting}[language=bash]
    jupyter notebook gen_2d_spectrum_example.ipynb
    \end{lstlisting}
    \item Follow the instructions and code cells to:
    \begin{itemize}
        \item Load output data
        \item Apply any window or filtering
        \item Compute the 2D Fourier transform
        \item Plot the 2D spectrum using Matplotlib
    \end{itemize}
\end{itemize}

This notebook is useful for adjusting analysis parameters and visually verifying results.


\end{document}